\usepackage[T1]{fontenc}	% ä,ü...
\usepackage[utf8]{inputenc} % utf-8 Unterstützung
\usepackage[ngerman]{babel} % Silbentrennung und Rechtschreibung Deutsch
\usepackage[{left=1.5cm,right=1.5cm,top=0cm,bottom=0cm}]{geometry} % Seitenränder Titelblatt
\usepackage[automark]{scrpage2} % Header und Footer

%%%%%%%%%%%%%%%%%%%%%%%
%% Packages
%%%%%%%%%%%%%%%%%%%%%%%
\usepackage{acronym} 		   % Für Abkürzungsverzeichnis
\usepackage{adjustbox} 		   % adjustbox, minipage..
\usepackage{amsmath}   		   % Allgemeine Matheumgebungen
\usepackage{amssymb}  		   % Fonts: msam,msbm, eufm & Mathesymbole, Mengen (lädt automatisch amsfonts)
\usepackage{array} 			   % Extending the array and tabular environment -> m,b,p,..
\usepackage{caption}  		   % Verändern der Schriftart von Bildunterschriften
\usepackage{changepage}
\usepackage{epstopdf}
\usepackage{expl3}
\usepackage{float}
\usepackage{framed, color}
\usepackage{graphbox}
\usepackage{graphicx}
\usepackage[colorlinks, linkcolor = black, citecolor = black, filecolor = black, urlcolor = blue]{hyperref}
\usepackage[none]{hyphenat}
\usepackage{listings}           % Erlaubt es Programmcode in der gewünschten Sprache zu hinterlegen (C++, Matlab,..)
\usepackage{longtable}
\usepackage{marginnote} 		% Für Seitenkommentare \marginnote
\usepackage{mathabx} 			% Package mit vielen weiteren Mathe Symbolen
\usepackage{mathtools}
\usepackage{mparhack}  			% Improved marginpar placement
\usepackage{multicol} 			% In­ter­mix sin­gle and mul­ti­ple columns
\usepackage{multirow} 			% Create tabular cells spanning multiple rows
\usepackage{paralist}
\usepackage{pdfpages} 
\usepackage{pxfonts} 			% Mathsymbols
\usepackage{rotating} 			% Rotation tools, including rotated fullpage floats
\usepackage[onehalfspacing]{setspace}
\usepackage{scrhack}        	% Fixes koma-script incompatibilities
\usepackage{subcaption}
\usepackage{tabularx}
\usepackage{textcomp} 			% Wird für Copyright-Symbol,Währungen, Musikalische-Symbole benötigt
\usepackage[most]{tcolorbox}
\usepackage{trfsigns}
\usepackage{varwidth}
\usepackage{wrapfig}

%%%%%%%%%%%%%%%%%%%%%%%
%% Setup Tikz
%%%%%%%%%%%%%%%%%%%%%%%
\usepackage{tikz}
\usepackage{struktex}
\usepackage{schemabloc}
\usepackage[normalem]{ulem}
\usetikzlibrary{circuits}
\usetikzlibrary{arrows}
\usetikzlibrary{circuits.ee.IEC}
\usetikzlibrary{patterns}
\usetikzlibrary{positioning}
\usetikzlibrary{shapes,arrows}

\usepackage{pgfplots}
\usepackage{pgfplotstable}
\pgfplotsset{compat=newest}

\tikzstyle{block} = [draw, rectangle, minimum height=3em, minimum width=4em]
\tikzstyle{input} = [coordinate]
\tikzstyle{output} = [coordinate]
\tikzstyle{pinstyle} = [pin edge={to-,thin,black}]
\tikzstyle{sum} = [draw, circle, node distance=1em, minimum height=1.5em]
\tikzset{>=latex}
\tikzset{%
	block/.style    = {draw, thick, rectangle, minimum height = 3em,
		minimum width = 3em},
	sum/.style      = {draw, circle, node distance = 1.5cm}, % Adder
	input/.style    = {coordinate}, % Input
	output/.style   = {coordinate} % Output
}
\newcommand\Umbruch[2][3cm]{\begin{varwidth}{#1}\centering#2\end{varwidth}}

%%%%%%%%%%%%%%%%%%%%%%%
% Caption Setup
%%%%%%%%%%%%%%%%%%%%%%%
\captionsetup[figure]{labelfont={it,bf},textfont={it}}
\captionsetup[table]{labelfont={it,bf},textfont={it},singlelinecheck=off,justification=centering}
\captionsetup[lstlisting]{labelfont={it,bf},textfont={it}}
\captionsetup[subfigure]{labelfont=bf,textfont=normalfont,singlelinecheck=off,justification=centering}

\newenvironment{nscenter}
{\parskip=-5pt\par\nopagebreak\centering}
{\parskip=-15pt\par\noindent\ignorespacesafterend}

%%%%%%%%%%%%%%%%%%%%%%%%
%% Header and Footer %%
%%%%%%%%%%%%%%%%%%%%%%%%
\pagestyle{scrheadings}
\renewcommand*{\sectionmarkformat}{}
\clearscrheadfoot
\ihead{\includegraphics[width = 3.5cm]{header/hsrlogo}}
\chead{\headmark}
%\ohead{\includegraphics[width = 3cm]{header/arendi_logo}}
\ifoot{\vspace{-0.25cm}\today}
\cfoot{\vspace{-0.5cm}\small{\Author}}
\ofoot{\vspace{-0.25cm}\pagemark}
\setheadsepline{.5pt}
\setfootsepline{.5pt}

%%%%%%%%%%%%%%%%%%%%%%%%%%%%%%%%%%%%%%%%
%% Textspacing %%
%%%%%%%%%%%%%%%%%%%%%%%%%%%%%%%%%%%%%%%%
\usepackage{titlesec}
\titlespacing{\section}{10pt}{0pt}{0.4em}
\titlespacing{\subsection}{10pt}{0pt}{0.2em}
\titlespacing{\subsubsection}{10pt}{0.2em}{0.2em}
\titlespacing{\paragraph}{10pt}{0.2em}{0.2em}

%%%%%%%%%%%%%%%%%%%%%
%% Load HSR colors %%
%%%%%%%%%%%%%%%%%%%%%
\usepackage{xcolor}
\usepackage{header/HSRColors}

%%%%%%%%%%%%%%%%%%
%% Bibliography %%
%%%%%%%%%%%%%%%%%%
\usepackage[backend=biber,style=ieee, defernumbers=true]{biblatex}
\addbibresource{Literatur.bib}


% Abbildungen im Quellenverzeichnis nach Alphabet ordnen, Rest nummerieren
%\DeclareFieldFormat{labelnumber}{\ifkeyword{abb}{\mknumalph{#1}}{#1}}

%%%%%%%%%%%%%%%%%%%%%%%%%%%%%%%%%%%
%% Itemize and Enumerate spacing %%
%%%%%%%%%%%%%%%%%%%%%%%%%%%%%%%%%%%
% \topsep: space between first item and preceding paragraph
% \partopsep: extra space added to \topsep when environment starts a new paragraph
% \itemsep: space between successive items. 
\usepackage{enumitem} % Controls Layout of itemize, enumerate, description
\setlist[itemize]{topsep=0pt,itemsep=-1ex,partopsep=1ex,parsep=1ex,after=\vskip0.1\baselineskip}
\setlist[enumerate]{topsep=0pt,itemsep=-1ex,partopsep=1ex,parsep=1ex,after=\vskip0.1\baselineskip}

%%%%%%%%%%%
%% Index %%
%%%%%%%%%%%
\usepackage{imakeidx}
\makeindex[intoc,columnseprule]
\indexsetup{firstpagestyle=plain}    % Show header/footer on index page

%%%%%%%%%%%%%%%%%%%%%%%
%% Aligned footnotes %%
%%%%%%%%%%%%%%%%%%%%%%%
\usepackage[hang]{footmisc}
\setlength{\footnotemargin}{1em}

%%%%%%%%%%%%%
%% Tabular %%
%%%%%%%%%%%%%
\newcolumntype{P}[1]{>{\raggedright\arraybackslash}p{#1}} % Tabelleninhalt linksausgerichtet
\newcolumntype{L}[1]{>{\raggedleft\arraybackslash}p{#1}} % Tabelleninhalt rechtsausgerichtet
\newcolumntype{C}[1]{>{\centering\arraybackslash}p{#1}} %  Tabelleninhalt zentriert

% If \Print=true, then make all links black for nicer print
\providecommand*{\True}{true}
\ifx \Print \True
\hypersetup{hidelinks}
\fi

\parindent0pt % Zeileneinzug verhindern
%%%%%%%%%%%%%%%%%%%%%%
%% Generelle Makros %%
%%%%%%%%%%%%%%%%%%%%%%
\newcommand{\matlab}[1]{\footnotesize{(Matlab: \texttt{#1})}\normalsize{}}

% Makro für Tabellenbilder gleich unterhalb der Linie
\newcommand\tabbild[2][]{%
	\raisebox{0pt}[\dimexpr\totalheight+\dp\strutbox\relax][\dp\strutbox]{%
		\includegraphics[#1]{#2}%
	}%
}

% Makro für Vorteile und Nachteil mit Plus und Minus
\newcommand\pro{\item[$+$]}
\newcommand\con{\item[$-$]}

%%%%%%%%%%
% Colors %
%%%%%%%%%%
\definecolor{black}{rgb}{0,0,0}
\definecolor{red}{rgb}{1,0,0}
\definecolor{white}{rgb}{1,1,1}
\definecolor{grey}{rgb}{0.8,0.8,0.8}
\definecolor{green}{rgb}{0,.8,0.05}
\definecolor{brown}{rgb}{0.603,0,0}
\definecolor{mymauve}{rgb}{0.58,0,0.82}
\definecolor{mygreen}{RGB}{28,172,0}
\definecolor{mygray}{rgb}{0.5,0.5,0.5}
\definecolor{mymauve}{rgb}{0.58,0,0.82}
\definecolor{mylilas}{RGB}{170,55,241}


%%%%%%%%%%%%%%%%%%%%%%%%%%%%
% Mathematical Operators %
%%%%%%%%%%%%%%%%%%%%%%%%%%%%
\DeclareMathOperator{\sinc}{sinc}
\DeclareMathOperator{\sgn}{sgn}
\DeclareMathOperator{\Real}{Re}
\DeclareMathOperator{\Imag}{Im}
\DeclareMathOperator{\euler}{e}
\DeclareMathOperator{\cov}{cov}
\DeclareMathOperator{\PolyGrad}{PolyGrad}
\DeclareMathOperator{\gradient}{grad}
\DeclareMathOperator{\rotation}{rot}
\DeclareMathOperator{\divergenz}{div}
\DeclareMathOperator{\imag}{j}

%Grösse Integral anpassen
\def\Int{\mbox{\Large$\displaystyle\int$\normalsize}}
\def\Int{\mbox{\Large$\displaystyle\iint$\normalsize}}
\def\OInt{\mbox{\Large$\displaystyle\oint$\normalsize}}

%Makro für 'd' von Integral- und Differentialgleichungen 
\newcommand*{\diff}{\mathop{}\!\mathrm{d}}

%%%%%%%%%%%%%%%%%%%%%%%%%%%
% Fouriertransform %
%%%%%%%%%%%%%%%%%%%%%%%%%%%

\unitlength1cm
\newcommand{\FT}
{
	\begin{picture}(1,0.5)
	\put(0.2,0.1){\circle{0.14}}\put(0.27,0.1){\line(1,0){0.5}}\put(0.77,0.1){\circle*{0.14}}
	\end{picture}
}


\newcommand{\IFT}
{
	\begin{picture}(1,0.5)
	\put(0.2,0.1){\circle*{0.14}}\put(0.27,0.1){\line(1,0){0.45}}\put(0.77,0.1){\circle{0.14}}
	\end{picture}
}

%%%%%%%%%%%%%%%
% Code Layout %
%https://en.wikibooks.org/wiki/LaTeX/Source_Code_Listings
%%%%%%%%%%%%%%%

\lstset{ %
	firstnumber=1,
	backgroundcolor=\color{white},   % choose the background color; you must add        \usepackage{color} or \usepackage{xcolor}
	basicstyle=\footnotesize,        % the size of the fonts that are used for the code
	breakatwhitespace=false,         % sets if automatic breaks should only happen at whitespace
	breaklines=true,                 % sets automatic line breaking
	captionpos=b,                    % sets the caption-position to bottom
	commentstyle=\color{mygreen},    % comment style
	deletekeywords={...},            % if you want to delete keywords from the given language
	otherkeywords={...},             % if you want to add more keywords to the set
	escapeinside={\%*}{*)},          % if you want to add LaTeX within your code
	extendedchars=true,              % lets you use non-ASCII characters; for 8-bits encodings only, does not work with UTF-8
	frame=single,	                 % adds a frame around the code
	keepspaces=true,                 % keeps spaces in text, useful for keeping indentation of code (possibly needs columns=flexible)
	keywordstyle=\color{blue},       % keyword style
	language=C++,                    % the language of the code   
	numbers=left,                    % where to put the line-numbers; possible values are (none, left, right)
	numbersep=5pt,                   % how far the line-numbers are from the code
	numberstyle=\tiny\color{mygray}, % the style that is used for the line-numbers
	rulecolor=\color{black},         % if not set, the frame-color may be changed on line-breaks within not-black text (e.g. comments (green here))
	showspaces=false,                % show spaces everywhere adding particular underscores; it overrides 'showstringspaces'
	showstringspaces=false,          % underline spaces within strings only
	showtabs=false,                  % show tabs within strings adding particular underscores
	stepnumber=2,                    % the step between two line-numbers. If it's 1, each line will be numbered
	stringstyle=\color{mymauve},     % string literal style
	tabsize=2,	                     % sets default tabsize to 2 spaces
	%title=\lstname                   % show the filename of files included with         \lstinputlisting; also try caption instead of title
}

\lstdefinestyle{customc++}{
	belowcaptionskip=1\baselineskip,
	%frame=L,
	xleftmargin=\parindent,
	language=C++,
	basicstyle=\footnotesize\ttfamily,
	keywordstyle=\bfseries\color{blue},
	commentstyle=\itshape\color{mygreen},
	identifierstyle=\color{black},
	stringstyle=\color{gray},
}

\lstset{ %
	texcl=true,
	firstnumber=1,
	backgroundcolor=\color{white},   % choose the background color; you must add  \usepackage{color} or \usepackage{xcolor}
	basicstyle=\footnotesize,        % the size of the fonts that are used for the code
	breakatwhitespace=false,         % sets if automatic breaks should only happen at whitespace
	breaklines=true,                 % sets automatic line breaking
	captionpos=b,                    % sets the caption-position to bottom
	commentstyle=\color{mygreen},    % comment style
	deletekeywords={...},            % if you want to delete keywords from the given language
	otherkeywords={...},             % if you want to add more keywords to the set
	escapeinside={\%*}{*)},          % if you want to add LaTeX within your code
	extendedchars=true,              % lets you use non-ASCII characters; for 8-bits encodings only, does not work with UTF-8
	frame=single,	                 % adds a frame around the code
	keepspaces=true,                 % keeps spaces in text, useful for keeping indentation of code (possibly needs columns=flexible)
	keywordstyle=\color{blue},       % keyword style
	language=Matlab,                    % the language of the code   
	numbers=left,                    % where to put the line-numbers; possible values are (none, left, right)
	numbersep=5pt,                   % how far the line-numbers are from the code
	numberstyle=\tiny\color{black}, % the style that is used for the line-numbers
	rulecolor=\color{black},         % if not set, the frame-color may be changed on line-breaks within not-black text (e.g. comments (green here))
	showspaces=false,                % show spaces everywhere adding particular underscores; it overrides 'showstringspaces'
	showstringspaces=false,          % underline spaces within strings only
	showtabs=false,                  % show tabs within strings adding particular underscores
	stepnumber=1,                    % the step between two line-numbers. If it's 1, each line will be numbered
	stringstyle=\color{mylilas},     % string literal style
	tabsize=2,	                     % sets default tabsize to 2 spaces
	%title=\lstname                   % show the filename of files included with         \lstinputlisting; also try caption instead of title
}

\lstdefinestyle{custommatlab}{
	belowcaptionskip=1\baselineskip,
	%frame=L,
	xleftmargin=\parindent,
	language=Matlab,
	basicstyle=\footnotesize\ttfamily,
	keywordstyle=\bfseries\color{blue},
	commentstyle=\itshape\color{mygreen},
	identifierstyle=\color{black},
	stringstyle=\color{mylilas},
}